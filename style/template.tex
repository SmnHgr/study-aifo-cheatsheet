%! Licence = CC BY-NC-SA 4.0

%! Author = mariuszindel
%! Date = 24. Jan 2021
%! Project = latex-test-template

\documentclass[a4paper, landscape, fontsize=5pt]{scrartcl} %-scrartcl = deutsche Sprache

% charset
\usepackage[T1]{fontenc}
\usepackage[utf8]{inputenc} %-inputenc = Umlaute möglich
\usepackage{ulem}

% use language german
\usepackage[ngerman]{babel} % with n is new spelling

% format page size
\usepackage{geometry}
\geometry{top=0.25cm,left=0.25cm,right=0.25cm,bottom=0.25cm}
%\textheight = 558pt

% tabular
\usepackage{tabularx}

% math
\usepackage{amsmath}
\usepackage{amssymb}
\usepackage{amsfonts}
\usepackage{enumitem}

% graphic
\usepackage{graphicx}
\graphicspath{{media/}}

% colors
\usepackage[dvipsnames]{xcolor}

% multi columns
\usepackage{multicol}

% make items compact
\setlist{topsep=0pt, leftmargin=3mm, nolistsep}
\setlength{\parindent}{0cm} % disable indention of text

% author and institute
\newcommand{\AUTHOR}{Marius Zindel }
\newcommand{\INSTITUTE}{Hochschule für Technik Rapperswil}

% define header and footer
\usepackage{fancyhdr}
\pagestyle{fancy}

%\fancyhead[RO]{\AUTHOR| \INSTITUTE}
%\fancyhead[LO]{\TITLE}
%\fancyfoot[RO]{\DELIVERYDATE}
%\fancyfoot[LO]{Created with \LaTeX}
\renewcommand\headrulewidth{0pt}
\renewcommand\footrulewidth{0pt}
%\headsep = -2pt
\footskip = 0pt

% define color
\definecolor{sectionColor}{HTML}{228B22}
\definecolor{subSectionColor}{HTML}{CB4154}
\definecolor{subSubSectionColor}{HTML}{FFFF00}
\definecolor{inlineCodeColor}{HTML}{FF00FF}
\definecolor{codeBackground}{RGB}{245,245,245}
\definecolor{gray}{rgb}{0.5,0.5,0.5}
\definecolor{darkGreen}{RGB}{0,150,0}
\definecolor{DarkPurple}{rgb}{0.4, 0.1, 0.4}

% define section format
\usepackage{sectsty}
\usepackage{titlesec}

\titleformat{name=\section}[block]{\sffamily\small}{}{0pt}{\colorsection}
\titlespacing*{\section}{0pt}{0pt}{0pt}
\newcommand{\colorsection}[1]{%
    \colorbox{sectionColor!40}{\parbox{0.98\linewidth}{\color{black}\thesection\ #1 }}} % 0.235\textwidth

% define subsection format
\titleformat{name=\subsection}[block]{\sffamily\small}{}{0pt}{\colorsubsection}
\titlespacing*{\subsection}{0pt}{0pt}{0pt}
\newcommand{\colorsubsection}[1]{%
    \colorbox{subSectionColor!40}{\parbox{0.98\linewidth}{\color{black}\thesubsection\ #1 }}}

% define subsubsection format
\titleformat{name=\subsubsection}[block]{\sffamily\small}{}{0pt}{\colorsubsubsection}
\titlespacing*{\subsubsection}{0pt}{0pt}{0pt}
\newcommand{\colorsubsubsection}[1]{%
    \colorbox{subSubSectionColor!50}{\parbox{0.98\linewidth}{\color{black}\thesubsubsection\ #1 }}}

% import code listings
\usepackage{listings}
\usepackage{beramono}
%! Licence = CC BY-NC-SA 4.0

%! Author = mariuszindel
%! Date = 22. Feb 2021
%! Project = latex-test-template

\definecolor{editorGray}{rgb}{0.95, 0.95, 0.95}
\definecolor{editorOcher}{rgb}{1, 0.5, 0} % #FF7F00 -> rgb(239, 169, 0)
\definecolor{editorGreen}{rgb}{0, 0.5, 0} % #007C00 -> rgb(0, 124, 0)
\definecolor{orange}{rgb}{1,0.45,0.13}
\definecolor{olive}{rgb}{0.17,0.59,0.20}
\definecolor{brown}{rgb}{0.69,0.31,0.31}
\definecolor{purple}{rgb}{0.38,0.18,0.81}
\definecolor{lightblue}{rgb}{0.1,0.57,0.7}
\definecolor{lightred}{rgb}{1,0.4,0.5}
\definecolor{stringstyle}{RGB}{0, 128, 0}

\lstdefinelanguage{CSS}{
    keywords={color,background-image:,margin,padding,font,weight,display,position,top,left,right,bottom,list,style,border,size,white,space,min,width, transition:, transform:, transition-property, transition-duration, transition-timing-function},
    sensitive=true,
    morecomment=[l]{//},
    morecomment=[s]{/*}{*/},
    morestring=[b]',
    morestring=[b]",
    alsoletter={:},
    alsodigit={-}
}

\lstdefinelanguage{JavaScript}{
    morekeywords={typeof, new, true, false, catch, function, return, null, catch, switch, var, if, in, while, do, else, case, break},
    morecomment=[s]{/*}{*/},
    morecomment=[l]//,
    morestring=[b]",
    morestring=[b]'
}

\lstdefinelanguage{HTML5}{
    language=html,
    sensitive=true,
    alsoletter={<>=-},
    morecomment=[s]{<!-}{-->},
    tag=[s],
    otherkeywords={
        % General
        >,
        % Standard tags
        <!DOCTYPE,
        </html, <html, <head, <title, </title, <style, </style, <link, </head, <meta, />,
        % body
        </body, <body,
        % Divs
        </div, <div, </div>,
        % Paragraphs
        </p, <p, </p>,
        % scripts
        </script, <script,
        % More tags...
        <canvas, /canvas>, <svg, <rect, <animateTransform, </rect>, </svg>, <video, <source, <iframe, </iframe>, </video>, <image, </image>, <header, </header, <article, </article
    },
    ndkeywords={
        % General
        =,
        % HTML attributes
        charset=, src=, id=, width=, height=, style=, type=, rel=, href=,
        % SVG attributes
        fill=, attributeName=, begin=, dur=, from=, to=, poster=, controls=, x=, y=, repeatCount=, xlink:href=,
        % properties
        margin:, padding:, background-image:, border:, top:, left:, position:, width:, height:, margin-top:, margin-bottom:, font-size:, line-height:,
        % CSS3 properties
        transform:, -moz-transform:, -webkit-transform:,
        animation:, -webkit-animation:,
        transition:,  transition-duration:, transition-property:, transition-timing-function:,
    }
}


\lstdefinestyle{Java}{
    language=java,
    backgroundcolor = \color{codeBackground},       %color for the background
    keywordstyle=\color{RoyalBlue}\ttfamily,        % style of keywords in source language
    stringstyle=\color{darkGreen}\ttfamily,         % style of strings in source language
    commentstyle=\color{DarkPurple!60}\ttfamily,    % style of comments in source language
    escapeinside={£}{£},                            % specify characters to escape from source code to LATEX
    showspaces=false,                               % emphasize spaces in code (true/false)
    showstringspaces=false,
    showtabs=false,                                 % emphasize tabulators in code (true/false)
    numbers=none,                                   % position of line numbers (left/right/none)
    numberstyle=\tiny\color{darkgray}\ttfamily,     % style used for line-numbers
    stepnumber=1,                                   % distance of line-numbers from the code
    tabsize=1,                                      % default tabsize
    breaklines=true,                                % automatic line-breaking
    breakatwhitespace=true,                         % sets if automatic breaks should only happen at whitespaces
%frame=single,                                   % showing frame outside code (none/leftline/topline/bottomline/lines/single/shadowbox)
    xleftmargin=0pt,
    xrightmargin=-3pt,
    frameround=tttt,                            % enable round corners
    rulecolor = \color{lightgray},              % Specify the colour of the frame-box
    captionpos = b                              % position of caption (t/b)
}

\lstdefinestyle{CSharp}{
    language=[Sharp]C,
    backgroundcolor = \color{codeBackground},       %color for the background
    keywordstyle=\color{RoyalBlue}\ttfamily,        % style of keywords in source language
    stringstyle=\color{darkGreen}\ttfamily,         % style of strings in source language
    commentstyle=\color{DarkPurple!60}\ttfamily,    % style of comments in source language
    escapeinside={£}{£},                            % specify characters to escape from source code to LATEX
    showspaces=false,                               % emphasize spaces in code (true/false)
    showstringspaces=false,
    showtabs=false,                                 % emphasize tabulators in code (true/false)
    numbers=none,                                   % position of line numbers (left/right/none)
    numberstyle=\tiny\color{darkgray}\ttfamily,     % style used for line-numbers
    stepnumber=1,                                   % distance of line-numbers from the code
    tabsize=1,                                      % default tabsize
    breaklines=true,                                % automatic line-breaking
    breakatwhitespace=true,                         % sets if automatic breaks should only happen at whitespaces
%frame=single,                                   % showing frame outside code (none/leftline/topline/bottomline/lines/single/shadowbox)
    xleftmargin=0pt,
    xrightmargin=-3pt,
    frameround=tttt,                            % enable round corners
    rulecolor = \color{lightgray},              % Specify the colour of the frame-box
    captionpos = b                              % position of caption (t/b)
}

\lstdefinestyle{JavaScript}{
    keywords={typeof, new, true, false, catch, function, return, null, catch, switch, var, if, in, while, do, else, case, break},
    ndkeywords={class, export, boolean, throw, implements, import, this},
    comment=[l]{//},
    backgroundcolor = \color{codeBackground},       %color for the background
    keywordstyle=\color{RoyalBlue}\ttfamily,        % style of keywords in source language
    stringstyle=\color{darkGreen}\ttfamily,         % style of strings in source language
    commentstyle=\color{DarkPurple!60}\ttfamily,    % style of comments in source language
    escapeinside={£}{£},                            % specify characters to escape from source code to LATEX
    showspaces=false,                               % emphasize spaces in code (true/false)
    showstringspaces=false,
    showtabs=false,                                 % emphasize tabulators in code (true/false)
    numbers=none,                                   % position of line numbers (left/right/none)
    numberstyle=\tiny\color{darkgray}\ttfamily,     % style used for line-numbers
    stepnumber=1,                                   % distance of line-numbers from the code
    tabsize=1,                                      % default tabsize
    breaklines=true,                                % automatic line-breaking
    breakatwhitespace=true,                         % sets if automatic breaks should only happen at whitespaces
%frame=single,                                   % showing frame outside code (none/leftline/topline/bottomline/lines/single/shadowbox)
    xleftmargin=0pt,
    xrightmargin=-3pt,
    frameround=tttt,                            % enable round corners
    rulecolor = \color{lightgray},              % Specify the colour of the frame-box
    captionpos = b                              % position of caption (t/b)
}

\lstdefinestyle{HTML}{
    language=HTML,
    backgroundcolor = \color{codeBackground},       %color for the background
    keywordstyle=\color{RoyalBlue}\ttfamily,        % style of keywords in source language
    stringstyle=\color{darkGreen}\ttfamily,         % style of strings in source language
    commentstyle=\color{DarkPurple!60}\ttfamily,    % style of comments in source language
    escapeinside={£}{£},                            % specify characters to escape from source code to LATEX
    showspaces=false,                               % emphasize spaces in code (true/false)
    showstringspaces=false,
    showtabs=false,                                 % emphasize tabulators in code (true/false)
    numbers=none,                                   % position of line numbers (left/right/none)
    numberstyle=\tiny\color{darkgray}\ttfamily,     % style used for line-numbers
    stepnumber=1,                                   % distance of line-numbers from the code
    tabsize=1,                                      % default tabsize
    breaklines=true,                                % automatic line-breaking
    breakatwhitespace=true,                         % sets if automatic breaks should only happen at whitespaces
%frame=single,                                   % showing frame outside code (none/leftline/topline/bottomline/lines/single/shadowbox)
    xleftmargin=0pt,
    xrightmargin=-3pt,
    frameround=tttt,                            % enable round corners
    rulecolor = \color{lightgray},              % Specify the colour of the frame-box
    captionpos = b                              % position of caption (t/b)
}

\lstdefinestyle{Python}{
    language=Python,
    backgroundcolor = \color{codeBackground},       %color for the background
    keywordstyle=\color{RoyalBlue}\ttfamily,        % style of keywords in source language
    stringstyle=\color{darkGreen}\ttfamily,         % style of strings in source language
    commentstyle=\color{DarkPurple!60}\ttfamily,    % style of comments in source language
    escapeinside={£}{£},                            % specify characters to escape from source code to LATEX
    showspaces=false,                               % emphasize spaces in code (true/false)
    showstringspaces=false,
    showtabs=false,                                 % emphasize tabulators in code (true/false)
    numbers=none,                                   % position of line numbers (left/right/none)
    numberstyle=\tiny\color{darkgray}\ttfamily,     % style used for line-numbers
    stepnumber=1,                                   % distance of line-numbers from the code
    tabsize=1,                                      % default tabsize
    breaklines=true,                                % automatic line-breaking
    breakatwhitespace=true,                         % sets if automatic breaks should only happen at whitespaces
%frame=single,                                   % showing frame outside code (none/leftline/topline/bottomline/lines/single/shadowbox)
    xleftmargin=0pt,
    xrightmargin=-3pt,
    frameround=tttt,                            % enable round corners
    rulecolor = \color{lightgray},              % Specify the colour of the frame-box
    captionpos = b                              % position of caption (t/b)
}

\lstdefinestyle{bash}{
    language=bash,
    backgroundcolor = \color{codeBackground},       %color for the background
    keywordstyle=\color{RoyalBlue}\ttfamily,        % style of keywords in source language
    stringstyle=\color{darkGreen}\ttfamily,         % style of strings in source language
    commentstyle=\color{DarkPurple!60}\ttfamily,    % style of comments in source language
    escapeinside={£}{£},                            % specify characters to escape from source code to LATEX
    showspaces=false,                               % emphasize spaces in code (true/false)
    showstringspaces=false,
    showtabs=false,                                 % emphasize tabulators in code (true/false)
    numbers=none,                                   % position of line numbers (left/right/none)
    numberstyle=\tiny\color{darkgray}\ttfamily,     % style used for line-numbers
    stepnumber=1,                                   % distance of line-numbers from the code
    tabsize=1,                                      % default tabsize
    breaklines=true,                                % automatic line-breaking
    breakatwhitespace=true,                         % sets if automatic breaks should only happen at whitespaces
%frame=single,                                   % showing frame outside code (none/leftline/topline/bottomline/lines/single/shadowbox)
    xleftmargin=0pt,
    xrightmargin=-3pt,
    frameround=tttt,                            % enable round corners
    rulecolor = \color{lightgray},              % Specify the colour of the frame-box
    captionpos = b                              % position of caption (t/b)
}

\lstdefinestyle{LaTeX}{
    language=TeX,
    backgroundcolor = \color{codeBackground},       %color for the background
    keywordstyle=\color{RoyalBlue}\ttfamily,        % style of keywords in source language
    stringstyle=\color{darkGreen}\ttfamily,         % style of strings in source language
    commentstyle=\color{DarkPurple!60}\ttfamily,    % style of comments in source language
    escapeinside={£}{£},                            % specify characters to escape from source code to LATEX
    showspaces=false,                               % emphasize spaces in code (true/false)
    showstringspaces=false,
    showtabs=false,                                 % emphasize tabulators in code (true/false)
    numbers=none,                                   % position of line numbers (left/right/none)
    numberstyle=\tiny\color{darkgray}\ttfamily,     % style used for line-numbers
    stepnumber=1,                                   % distance of line-numbers from the code
    tabsize=1,                                      % default tabsize
    breaklines=true,                                % automatic line-breaking
    breakatwhitespace=true,                         % sets if automatic breaks should only happen at whitespaces
%frame=single,                                   % showing frame outside code (none/leftline/topline/bottomline/lines/single/shadowbox)
    xleftmargin=0pt,
    xrightmargin=-3pt,
    frameround=tttt,                            % enable round corners
    rulecolor = \color{lightgray},              % Specify the colour of the frame-box
    captionpos = b                              % position of caption (t/b)
}



\lstdefinestyle{htmlcssjs} {
    language=HTML5,
    alsolanguage=JavaScript,
    alsodigit={.:;},
% General design
%  backgroundcolor=\color{editorGray},
    frame=none,
% line-numbers
% xleftmargin={0.75cm},
    numbers=none,
% Code design
    identifierstyle=\color{black},
    backgroundcolor = \color{codeBackground},       %color for the background
    keywordstyle=\color{RoyalBlue}\ttfamily,        % style of keywords in source language
    ndkeywordstyle=\color{editorGreen}\bfseries,
    stringstyle=\color{darkGreen}\ttfamily,         % style of strings in source language
    commentstyle=\color{DarkPurple!60}\ttfamily,    % style of comments in source language
% Code
    tabsize=1,
    showtabs=false,
    showspaces=false,
    showstringspaces=false,
    extendedchars=true,
    breaklines=true,
}



\lstset{style=htmlcssjs}
\lstset{aboveskip=0pt, belowskip=0pt}
\lstset{basicstyle={\footnotesize\ttfamily}}
\lstset{
    literate=  % Allow for German characters in lstlistings.
        {Ö}{{\"O}}1
        {Ä}{{\"A}}1
        {Ü}{{\"U}}1
        {ü}{{\"u}}1
        {ä}{{\"a}}1
        {ö}{{\"o}}1
}

% dotted rule
\usepackage{dashrule}
\usepackage{tikz}
\usetikzlibrary{decorations.markings}
\newcommand{\drule}[3][0]{
    \tikz[baseline]{\path[decoration={markings,
    mark=between positions 0 and 1 step 2*#3
    with {\node[fill, circle, minimum width=#3, inner sep=0pt, anchor=south west] {};}},postaction={decorate}]  (0,#1) -- ++(#2,0);}}

% no indentation
\setlength{\parindent}{0cm}

% include lorem ipsum
\usepackage{lipsum}

\setlist[itemize]{noitemsep, topsep=0pt}

%TODO
\newcommand{\code}{\lstinline[keywordstyle=\color{inlineCodeColor}, basicstyle=\color{inlineCodeColor}, directivestyle=\color{inlineCodeColor}, stringstyle=\color{inlineCodeColor}, identifierstyle=\color{inlineCodeColor}]}


